\title{Summary: Monte Carlo Variance Reduction for Multi-physics Processes in Moving Systems}
\author{Chelsea A. D'Angelo}
\documentclass[12pt]{article}
%\pagestyle{empty}

\begin{document}
\maketitle

The rapid design iteration process of complex nuclear systems has long been
aided by computational simulation.  Traditionally, these simulations
involve radiation transport in static geometries.  However, in certain
scenarios, it is desirable to investigate dynamic systems and the effects caused
by the motion of one or more components.  
For example, fusion energy systems (FES) are purposefully designed with modular components that can be moved in and
out of a facility after shutdown for maintenance purposes.  
  It is particularly important to accurately quantify 
the shutdown dose rate (SDR), caused by the photons emitted by activated structural materials,
during maintenance procedures that may cause facility personnel to be in closer
proximity to activated equipment.
This type of analysis requires neutron transport to determine the neutron flux,
activation analysis to determine the isotopic inventory, and finally a 
photon transport calculation to determine the SDR.

While Monte Carlo (MC) calculations are considered to be the most accurate method for simulating
radiation transport, the computational expense of obtaining results with low
error 
in systems with heavy shielding can be prohibitive.  
However, variance reduction (VR)
methods can be used to increase the computational efficiency.  
There are several types
of VR methods, but the basic theory is to artificially increase the simulation of
events that will contribute to the quantity of interest such as flux or dose
rate. 
 One class of VR techniques takes advantage of a deterministic estimate of the adjoint 
solution of the transport equation to automatically
generate biasing parameters to accelerate the MC transport. 
The adjoint flux has significance as the importance of a region of
phase space to the objective function.

One hybrid VR technique used to optimize the initial transport step of a
multi-step process is known as 
the Multi-Step Consistent Adjoint Driven Importance Sampling
(MS-CADIS). 
 The basis of MS-CADIS is that the importance (adjoint) function used
in each step of the problem must represent the importance of the particles to
the final objective function.  
%As the spatial configuration of the materials
%changes, the probability that they will contribute to the objective function
%also changes.
In the specific case of SDR calculations, the importance function for the neutron transport step
must capture the probability of materials to become activated and subsequently emit photons that
will make a significant contribution to the SDR.
The Groupwise Transmutation (GT)-CADIS method 
is an implementation of MS-CADIS
that optimizes the neutron transport step of SDR calculations.
GT-CADIS generates an adjoint
neutron source based on certain assumptions and approximations about the
transmutation network.  
This source is used for adjoint transport and the resulting flux is used to
generate the biasing parameters to optimize the forward neutron transport.
For cases involving coupled multi-physics analysis in dynamics systems, such as SDR calculations
during maintenance activities,
a new hybrid deterministic/MC VR technique that extends 
GT-CADIS for dynamic systems by calculating a time-integrated adjoint
neutron source is currently in development.

The successful completion of this project will demonstrate the efficacy of a workflow and tools
necessary to efficiently calculate quantities of interest resulting from
coupled, multi-physics processes in dynamic systems.  
The driving force behind this work is the quantification of the
SDR resulting from the coupled neutron
irradiation-photon emission that occurs in FES;
specifically investigating how to optimize the SDR calculation when %changes as a function of time when
activated system components are moved during maintenance activities.
Tools have been constructed to implement 
rigid-body transformations on CAD-based geometry that serves as input for both
deterministic and Monte Carlo radiation transport.
Other tools have been developed to collect and manipulate the adjoint
flux data from each time step of geometry movement after shutdown which is
ultimately used to create VR parameters to optimize the forward neutron transport.

\end{document}

%
% Work and Research Experience Resume
% 

\section{Work \& Research Experience} \vspace{-2mm} 
\normalsize
%\begin{tabular}{r|p{15cm}}	
\begin{tabular}{r|p{13.7cm}}	

% --------------------------------------------------------------------------------------
% UW RA
% --------------------------------------------------------------------------------------
\multicolumn{1}{c}{} \vspace{-1mm} \\   % Small line skip
\multicolumn{2}{l}{\hspace{35mm} \large {\fontfamily{ptm}\selectfont {\bf University of Wisconsin - Madison}}, \footnotesize 1500 Engineering Dr., Madison, Wisconsin, 53705} 
\vspace{2mm}\\
   \textsc{Aug 2012 - Present}       & \textbf{Graduate Research Assistant:}
   \textit{Computational Nuclear Engineering Research Group}\\%   \textit{Nuclear Engineering and Engineering Physics} \\
		 & \small{ \vspace{-2.0mm} 
\begin{itemize}[leftmargin=4mm] 
%   	\item Patented a segmented reconstruction technique for artifact reduction in MRI [Pat: \ref{lqsm_patent}]
%	\item Collaborated with medical doctors and other clinical researchers to secure NIH grant funding
%	\item Submitted abstracts and presented findings at various international conferences [Pubs: \ref{ismrm_2016_paper},\ref{cmrr_2015_paper},\ref{ismrm_2014_paper}]
  \item Thesis topic: Development of an automated Monte Carlo variance reduction
        technique for multi-physics processes occurring in dynamic systems.  The
        specific use case is the optimization of the neturon transport step
        of shutdown dose rate analysis of fusion energy systems that undergo
        geometry movement after shutdown. [Pub. \ref{ans_2018}]
  \item Integrating the GT-CADIS variance reduction method into a
        user-friendly workflow in the Python for Nuclear Engineering (PyNE) toolkit
  \item Developed topology restoration tool to prepare polygon-mesh
	  computational phantoms for radiation transport simulations [Pub.
		\ref{ans_2017}]
  \item Collaborated with NASA to perform Fluka simulations of radiation environment on Mars
  \item Performed 3D neutronics analysis of the ARIES-ACT2 experimental fusion
	  energy device [Pub. \ref{aries}]
  \item Compared unstructured mesh capabilities of MCNP6 and DAGMCNP [Pub. \ref{ans_2013}]
%  \item Presented development work and findings at several conferences
 \end{itemize} 
 \vspace{-4.5mm}   % Spacing that looked nice...
} \\ 
% --------------------------------------------------------------------------------------
% Los Alamos National Laboratory
% --------------------------------------------------------------------------------------
\multicolumn{1}{c}{} \vspace{2mm} \\   % This little doohickey eliminates the default | before Los Alamos National Laboratory
\multicolumn{2}{l}{\hspace{35mm} \large {\fontfamily{ptm}\selectfont {\bf Los Alamos National Laboratory}}, \footnotesize P.O. Box 1663, Los Alamos, New Mexico, 87545}
\vspace{2mm}\\
 
% --------------------------------------------------------------------------------------
% (2) Post-Bachelors W-13
% --------------------------------------------------------------------------------------
\textsc{May 2011 - July 2012} & \textbf{Post-Bachelor's/Graduate Research Assistant:}\ \textit{W-13: Advanced Engineering Analysis} \\
    & \small{ \vspace{-2.0mm} 
	\begin{itemize}[leftmargin=4mm]

	 %\item Assisted in the integration of radiation transport with finite element analysis on weapons systems
          \item Tested new features of the unstructured mesh capability of
		  MCNP6 [Pub. \ref{mcnp6_um}]
	  \item Created training material for generating unstructured mesh
		  models with Abaqus/CAE 
	  \item Performed radiation transport analysis on unstructured mesh models of weapons systems
	  \item Assisted with experiment setup and maintenance  and performed
		  MCNP6 calculations in support of experiments in the Ion Beam
			Materials Lab [Pub. \ref{mst1} and \ref{mst2}]
 	  \item Obtained Department of Energy Q-level security clearance %and Sigmas 1-10,11,12,13,15 

	\end{itemize}
 \vspace{-2.0mm}  
 } \\
 
% --------------------------------------------------------------------------------------
% (3) UGA XCP-3
% --------------------------------------------------------------------------------------
  \textsc{May 2010 - Aug 2010}  & \textbf{Undergraduate Intern:} \textit{XCP-3: Monte Carlo Codes} \\
      & \small{ \vspace{-2.0mm} 
 \begin{itemize}[leftmargin=4mm]
	\item Created benchmark-type problems for verification and
		  validation of the use of Abaqus/CAE unstructured mesh 
		  geometries with MCNP6 [Pub. \ref{uga}]
 
 \end{itemize}
 \vspace{-4.5mm}
 } \\



\end{tabular}


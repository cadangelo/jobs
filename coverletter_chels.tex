%\resumesection{Cover Letter} 

\begin{center}
\begin{minipage}{\textwidth}

  \normalsize

	\vspace{2mm}
  
  \begingroup
  \leftskip.65in
  \rightskip\leftskip

W-13 Weapons Systems Analysis\\
Los Alamos National Laboratory \\
P.O. Box 1663 \\
Los Alamos, NM 87545
  
  \vspace{4mm} 

% Introduction  / strengths
With research experience in modeling complex geometries and developing
software for radiation transport simulation and analysis, I believe I am
well-suited for the Engineering Analyst position in W-13
%, IRC57826,
at Los Alamos National Laboratory.
For your consideration, I have submitted a full r\'{e}sum\'{e} along with
the following summary of my skills and
experience relevant to this position. \\


\begin{itemize}[leftmargin=.875in,rightmargin=.875in,itemsep=1.0mm]

%\item Programming experience using Python, Matlab, and C/C++ for a variety of physics and data analysis applications
%\item Nine years working on Unix and Linux operating systems, such as Red Hat
%\item Experience running simulations using the high-performance computing systems at LANL
%\item Strong desire in continuing a career at LANL focused on weapons systems and national security issues
%\item Developed models and performed radiation transport simulations in \textsc{MCNP}, and cross-validated results with experimental data
%\item Familiarity with multi-physics analysis of nuclear weapons and working with Q-level sensitive material
%\item Effective at communicating results on advanced scientific topics, demonstrated through a history of writing publications and presenting at conferences
%\item Fourier Analysis for reconstruction of 
%\item Experienced with several physics-based codes, including radiation transport in \textsc{MCNP} and finite-element modeling and development in Abaqus/CAE
%\item Multiple years of experience with the high-performance computing systems at LANL
%\item Strong willingness to learn new skills, further my education, and excel in my current area of research 
               \item Bachelor's degree in Chemical Engineering, Master's degree
		       and future Ph.D. in Nuclear Engineering
                \item Over six years of experience with MCNP analysis of
			mesh-based geometries of complex nuclear
		      systems including the ARIES-ACT2 fusion energy device
		      [Pub. \ref{aries}],
		      computational human phantoms
		      [Pub. \ref{ans_2017}], weapons systems, and material testing
		      experiments in the
		      Annular Core Research Reactor (ACRR) [Pub.
		      \ref{mcnp6_um}]
	       \item Graduate research has revolved around the development of
		       computational tools 
		       %for Monte Carlo analysis and
		       and generation of models for Monte Carlo radiation transport
		       simulations
		       in high performance computing environments
	       \item Thesis work involves automated variance reduction for
		       coupled multi-physics processes occurring in dynamic
		       systems
              \item Held DOE Q-level security clearance from September 2011 to July
		      2012 and submitted for re-investigation February 2017
		\item  Have worked on projects funded by DOE at both LANL and
			the University of Wisconsin (UW) and my thesis work is supported by the Office of Fusion Energy
		\item Effective member of research teams at LANL and
			UW, brainstorming approaches to problems, dividing the
			work, and then independently fulfilling my duties to the teams and
			customers by ensuring timely execution of deliverables
		\item Spent several months working alongside a researcher
			at LANL's Ion Beam Materials Laboratory
			on beamline validation experiments 
		        [Pubs. \ref{mst1} and \ref{mst2}]
                        as well as
			visiting ACRR to interface with engineers performing
			materials testing to inform the MCNP simulations
			supporting this experimental work
	       \item Involved in several analysis projects
		      [Pubs. \ref{nair}, \ref{aries}]	       
			       that require the hand calculation of material compositions
			     and geometric quantities to support complex Monte
			      Carlo radiation transport simulations
	       \item  LANL Undergraduate Student project centered around validation and
		       verification of the MCNP6 unstructured mesh
		       capability [Pub. \ref{uga}]
	       \item Experienced in writing scripts to parse MCNP output
	               and plot results as well as writing scripts to
		       prepare mesh geometries for use with DAGMC
		       radiation transport simulations
                \item Participated in weekly project meetings with external collaborators (NASA and
	              ITER) to develop timelines, review preliminary results, and discuss
		      solutions to technical challenges
		\item Wrote formal reports to summarize project
			deliverables to external customers as well as conference
			papers and presentations to clearly demonstrate
			utility of software
			developments and analysis results
	       \item Use conferences and meetings as a means to lead
		       discussions with current and potential customers about
		       the software that my team and I have developed and I look forward to the opportunity to travel to
		       more of these events to network with other
			scientists and expand my knowledge
		       


\end{itemize} 

\vspace{5mm} % Go-the extra mile

%With a wide range of experiences at Los Alamos National Laboratory, I look forward to helping advance the engineering capabilities in 

\vspace{2mm}

%Attached to this application, you will find my r\'{e}sum\'{e} for your review. I appreciate your consideration and thank you for your time.
After completing my doctoral program, it would be a pleasure 
to continue my career working on weapons systems for
national security applications at LANL.  
I appreciate your consideration and look forward to hearing from you. 
  \vspace{6mm}
  
  Sincerely, \\ \vspace{-4mm}
  
%  \begin{figure}[h!]
 %\hspace{.7in} \includegraphics[width=1.5in]{../Signature/signature.png}
% \hspace{.7in} \includegraphics[width=1.5in]{signature.pdf}
%  \end{figure}
  
  Chelsea D'Angelo
  
  \endgroup
  
%  \endgroup
  \end{minipage}
\end{center}

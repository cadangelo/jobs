%\resumesection{Cover Letter} 

\begin{center}
\begin{minipage}{\textwidth}

  \normalsize

	\vspace{1mm}
  
  \begingroup
  \leftskip.65in
  \rightskip\leftskip

\today \\
W-13 Weapons Systems Analysis\\
Los Alamos National Laboratory \\
%P.O. Box 1663 \\
%Los Alamos, NM 87545
  
  \vspace{2mm} 

% Introduction  / strengths
With research experience in modeling complex geometries and developing
software for radiation transport simulation and analysis, I believe I am
well-suited for the R\&D Engineer 2  position in W-13
%, IRC57826,
at Los Alamos National Laboratory.
For your consideration, I have submitted a full r\'{e}sum\'{e}, brief summary of my thesis research, and
the following list of skills and
experience relevant to this position. \\


\begin{itemize}[leftmargin=.875in,rightmargin=.75in,itemsep=1.0mm]
               \item Bachelor's degree in Chemical Engineering, Master's degree
		       and future Ph.D. in Nuclear Engineering
               \item U.S. citizen with active DOE Q-level security clearance 
                \item Over seven years of experience with MCNP analysis of
			mesh-based geometries of complex nuclear
		      systems including the ARIES-ACT2 fusion energy device
		      [Pub. \ref{aries}],
		      computational human phantoms
		      [Pub. \ref{cp2017}], weapons systems, and material testing
		      experiments in the
		      Annular Core Research Reactor [Pub.
		      \ref{mcnp6_um}]
               \item Performed majority of the radiation transport simulations 
                     in massively parallel, high performance computing 
                     environments at both LANL (Turing, Redtail, Yellowtail) 
                     and the University of Wisconsin 
	       \item Involved in the introduction of the unstructured mesh capability
                     of MCNP6 as a student in LANL's Monte Carlo Methods, 
                     Codes, and Applications Group (XCP-3)
                     creating benchmark-type problems for validation and
		     verification [Pub. \ref{uga}] and 
                     in the Advanced Engineering Analysis Group (W-13) 
                     testing new features and 
                     presenting the capability at the Simulia Community Conference
                     [Pub. \ref{mcnp6_um}]
	      % \item Graduate research has involved the development of
	%               computational tools
	%	       %for Monte Carlo analysis and
	%	       and generation of models for Monte Carlo radiation transport
	%	       simulations
	%	       in high performance computing environments [Pub. \ref{ans_2017}]
	       \item Ph.D. thesis work involves the development of an automated variance reduction technique for
		       coupled multi-physics processes occurring in 
		       systems that undergo geometry movement
                       [Pub. \ref{ans_2018}].  
                       Quantifying the efficiency of this technique will require
                       statistical analysis of the error in the transport calculations.
              \item Experience with radiation transport in PARTISN, specifically
                    for generation of the adjoint flux used in adjoint-driven 
                    variance reduction techniques
              \item Use Trelis (Cubit) and Abaqus/CAE to generate solid and 
                    mesh models for radiation transport %with DAGMC and MCNP6
	      %\item  Have worked on projects funded by DOE at both LANL and
	      %		the University of Wisconsin (UW) and my thesis work is supported by the Office of Fusion Energy
		\item Worked alongside a researcher
			at LANL's Ion Beam Materials Laboratory
			on beamline validation experiments 
		        [Pubs. \ref{mst1}, \ref{mst2}]
                        and
			visited ACRR to interface with engineers performing
        			materials testing to inform the MCNP simulations
			supporting the experimental work
	       \item Involved in several analysis projects
		      [Pubs. \ref{nair}, \ref{aries}]	       
		     that require the hand calculation of material compositions
		     and geometric quantities to support complex Monte
		     Carlo radiation transport simulations
	       \item Experienced in writing scripts to parse MCNP output
	               and plot results as well as use VisIt and Abaqus/CAE for 3D visualization
               \item Developed algorithm and tools to 
		       prepare surface mesh geometries for DAGMC
		       radiation transport simulations [Pub. \ref{ans_2017}]
		\item Effective member of research teams at LANL and
		      UW that have included external customers (ITER and NASA) 
                      and a variety of engineers, physicists, and CAD-specialists.
                      Work together to brainstorm approaches to problems.
                      Independently
                      ensure timely execution of deliverables including
                      mesh-models prepared for radiation transport
                      and simulation results.
                %\item Participated in weekly project meetings with external 
                %      collaborators (NASA and ITER) to develop timelines, discuss
		%      solutions to technical challenges
	%	\item Wrote formal reports to summarize project
	%		deliverables to external customers as well as conference
	%		papers and presentations to clearly demonstrate
	%		utility of software
	%		developments and analysis results
               \item Take pride in preparing engaging oral presentations and 
                     effectively explaining methods and results in 
                     formal reports.  Have been 
                     complimented on concise, clear
                     style of writing by advisor and peers. 
	       \item Use conferences as means to lead
		       discussions with current and potential customers about
		       software that my team and I have developed. Look forward to traveling to
		       more of these events to network with other
			scientists and expand my knowledge.
		       


\end{itemize} 

\vspace{4mm} % Go-the extra mile

%With a wide range of experiences at Los Alamos National Laboratory, I look forward to helping advance the engineering capabilities in 

%\vspace{1.5mm}

%Attached to this application, you will find my r\'{e}sum\'{e} for your review. I appreciate your consideration and thank you for your time.
After completing my doctoral program, it would be a pleasure 
to continue my career working on complex nuclear systems for
national security applications at LANL.  
I appreciate your consideration and look forward to hearing from you. 
  \vspace{4mm}
  
  Sincerely, \\ \vspace{-4mm}
  
%  \begin{figure}[h!]
 %\hspace{.7in} \includegraphics[width=1.5in]{../Signature/signature.png}
% \hspace{.7in} \includegraphics[width=1.5in]{signature.pdf}
%  \end{figure}
  
  Chelsea D'Angelo
  
  \endgroup
  
%  \endgroup
  \end{minipage}
\end{center}
